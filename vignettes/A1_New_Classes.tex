%\appendix

\renewcommand\thefigure{\thesection.\arabic{figure}}
\renewcommand\thetable{\thesection.\arabic{table}}

\section{New \texttt{FLR - S4} classes}

\setcounter{figure}{0} 
\setcounter{table}{0}

\subsection{\texttt{FLBDsim} class} 

	\texttt{FLBDsim} class has been created in order to facilitate the simulation of population growth
	in populations aggregated in biomass, i.e. $g(.)$ in Equation~\ref{eq:BDPG}. The population dynamics are 
	simulated as follows: 
	
			\begin{equation}
				B_{y,s} = B_{y0,s0} + g(B_{y0,s0})\cdot\varepsilon_{y,s} - C_{y0,s0}
			\end{equation}
	
	\noindent where $B$ is the biomass, $C$ the catch, $y0$ and $s0$ are the subscripts of previous season's year and season
	and $\varepsilon$ is the uncertainty value in year $y$ and season $s$.    
	It is a \texttt{S4} class and has 10 slots:
	
	
\begin{description}
	\item[\texttt{name, desc, range}:] Slots common to all \texttt{FLR} objects. 
	\item[\texttt{model}:] \texttt{Character string} or \texttt{formula}. If character, it must coincide with an already existing growth model.
			If formula, the parameters must be slots in the object or elements of \texttt{covar} slot. Currently, there is only one 
			growth model available, \texttt{'PellaTom'} that corresponds with Pella-Tomlinson growth model \citep{Pella1969}.					
	\item[\texttt{biomass}:] \texttt{FLQuant} to store biomass in weight. The dimension in \texttt{quant, unit} and \texttt{area}
			must be equal to 1 and in the rest of the dimensions it must be congruent with general simulation settings. 
	\item[\texttt{catch}:] \texttt{FLQuant} to store total catch in weight. The dimension in \texttt{quant, unit} and \texttt{area}
			must be equal to 1 and in the rest of the dimensions it must be congruent with general simulation settings. 
	\item[\texttt{uncertainty}:] \texttt{FLQuant} to store the error that is multiplied to the point estimate of growth . 
			The dimension in \texttt{quant, unit} and \texttt{area}
			must be equal to 1 and in the rest of the dimensions it must be congruent with general simulation settings.
			Thus, a different error can be used for each year, season and iteration. 
	\item[\texttt{params}:] An \texttt{array} to store the parameters of the model. The dimensions of the array are 
			\texttt{params, year, season, iter}. The dimension in  \texttt{year, season} and \texttt{iter} must be 
			congruent with general simulation settings. Thus, a different set of parameters can be used for each year, 
			season and iteration.
	\item[\texttt{covar}:] An \texttt{FLQuants} object. The elements of the list are used to store covariates' values and it is used
			to apply growth models with covariates. Its functionality is the same as in \texttt{FLSR} object.
	% \item[\texttt{alpha}:] A numeric value bigger than one, which indicates (in percentage) how big can be the biomass in comparison with the carriying capacity.
	\item[\texttt{alpha}:] An array with dimension \texttt{[year = ny, season = ns, iteration = ni]} 
	  with  year, season and iteration dependent value bigger than one which indicates, in percentage, how big can be the biomass in comparison with the carrying capacity.
\end{description}


\subsection{\texttt{FLSRSim} class}

\texttt{FLSRsim} class has been created in order to facilitate the simulation of recruitment in age structured 
	 populations. The recruitment dynamics are simulated as follows:  
 
 \begin{equation}\label{eq:SRR}
    R_{y,s} = \Phi(S_{y-tl_0,s-tl_1},covars_{y-tl_0,s-tl_1})\cdot \varepsilon_{y,s} \cdot \rho_{y,s}
 \end{equation}
 
 \noindent where $R_{y,s}$ is the recruitment in year $y$ and season $s$, $\Phi$ is the stock-recruitment model, 
 $tl_0$ and $tl_1$ are the year and season lag between spawning and recruitment, respectively, $S_{y-tl_0,tl_1}$ and
 $covars_{y-tl_0,s-tl_1}$ are the stock index and covariates in year $y-tl_0$ and season $tl_1$,
 $\varepsilon_{y,s}$ is the uncertainty value in year $y$ and season $s$ and $\rho_{y,s}$ is the 
 proportion of recruitment that recruits in year $y$ and season $s$ and is produced
 by stock index $S$ in year $y-tl_0$ and season $tl_1$ .

\begin{description}
  \item[\texttt{name, desc, range}:] Slots common to all \texttt{FLR} objects. 
  \item[\texttt{rec}:] An \texttt{FLQuant} with dimension $[1,ny,1,ns,1,ni]$ used to store recruitment.
	\item[\texttt{ssb}:] An \texttt{FLQuant} with dimension $[1,ny,1,ns,1,ni]$ used to store SSB or the
      stock index used in the stock-recruitment relationship.
	\item[\texttt{covar}:] An \texttt{FLQuants} to store the covariates used in the stock-recruitment relationship.
      For details on the use of this slot look at the description of \texttt{FLSR} class.
	\item[\texttt{uncertainty}:]  An \texttt{FLQuant} with dimension $[1,ny,1,ns,1,ni]$ used to store the uncertainty
      related to stock-recruitment process. The content of this slot is multiplied to the point estimate of recruitment.
      As its effect is multiplicative, then set it equal to 1 for all year, season and iteration if uncertainty is not going to be
      considered around stock-recruitment curve.
	\item[\texttt{proportion}:] An \texttt{FLQuant} with dimension $[1,ny,1,ns,1,ni]$ used to store the proportion of the recruitment produced by stock index in year $y-\texttt{timelag}[1,s]$ and season $\texttt{timelag}[2,s]$ that recruits in year $y$ and season $s$.\\
      The content of this slot is multiplied to the point estimate of recruitment.
      As its effect is multiplicative, then set it equal to 1 if all the recruitment produced by certain stock index is recruited at the same time and set it equal to 0 if  none of the recruitment produced by certain stock index is  recruited in that season.
	\item[\texttt{model}:] \texttt{Character string} or \texttt{formula}. If character it specifies the name of the function used to simulate the recruitment. 
       If formula  the left hand side of $\sim$ must be equal to \texttt{rec} and the elements in right hand side must be 
       among \texttt{ssb, covars} and \texttt{params}.
	\item[\texttt{params}:]	An array with dimension $\texttt{[nparams, ny, ns, ni]}$, 
		thus, the parameters may be year, season and iteration dependent.   
	  Year dimension in parameters may be useful to model regime shifts.
	\item[\texttt{timelag}:] A matrix with dimension $\texttt{[2,ns]}$. This object indicates the time lag between spawning and recruitment in each season.
        For each season, the element in the first row indicates the age at recruitment and the element in 
        the second row indicates the season at which the recruitment was spawn.
\end{description}
