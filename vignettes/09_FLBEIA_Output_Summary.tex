
\section{Output summary} \label{sec:FLBEIAoutput}

  There are several functions available to summarise the results of the FLBEIA output.
  
  There are two types of functions: 
  
\subsection{Summary functions}

  Functions that summarise the results of the FLBEIA output in data frames.
  
  \begin{description}

    \item[\texttt{summary\_flbeia}:] An array with four dimensions: stock, year, iteration, indicator. 
      The indicators are: recruitment, ssb, f, biomass, catch, landings and discards.

    \item[\texttt{B\_flbeia}:] Biomass values by stock. 
      An array with three dimensions: stock, year and iteration.
    \item[\texttt{F\_flbeia}:] Fishing mortality values by stock. 
      An array with three dimensions: stock, year and iteration.
    \item[\texttt{SSB\_flbeia}:] Spawning stock biomass values by stock. 
      An array with three dimensions: stock, year and iteration.
    \item[\texttt{R\_flbeia}:] Recruitment values by stock. 
      If the stock follows a biomass dynamics, then this function gives the growth. 
      An array with three dimensions: stock, year and iteration.
    
    \item[\texttt{C\_flbeia}:] Catches by fleets and stock. 
      An array with three dimensions: stock, year and iteration.
    \item[\texttt{L\_flbeia}:] Landings by fleets and stock. 
      An array with three dimensions: stock, year and iteration.
    \item[\texttt{D\_flbeia}:] Discards by fleets and stock. 
      An array with three dimensions: stock, year and iteration.

    \item[\texttt{advSum, advSumQ}:] Data frame with the indicators related with the management advice (TAC). 
      The indicators are: catch, discards, discRat, landings, quotaUpt and tac.
    \item[\texttt{bioSum, bioSumQ}:] Data frame with the biological indicators. 
      The indicators are: biomass, catch, catch.iyv, discards, disc.iyv, f, landings, land.iyv, rec and ssb.
    \item[\texttt{fltSum, fltSumQ}:] Data frame with the indicators at fleet level. 
      The indicators are: capacity, catch, costs, discards, discRat, effort, fcosts, gva, income, landings, netProfit, 
      nVessels, price, profits, quotaUpt, salaries, vcosts and profitability.
    \item[\texttt{fltStkSum, fltStkSumQ}:] Data frame with the indicators at fleet and stock level. 
      The indicators are: landings, discards, catch, price, quotaUpt, tacshare, discRat and quota.
    \item[\texttt{npv}:] A data frame with the net present value per fleet over the selected range of years.
    \item[\texttt{mtSum, mtSumQ}:] Data frame with the indicators at fleet. 
      The indicators are: effshare, effort, income and vcost.
    \item[\texttt{mtStkSum, mtStkSumQ}:] Data frame with the indicators at fleet and metier level. 
      The indicators are: catch, discards, discRat, landings and price.
    \item[\texttt{riskSum}:] A data frame with the risk indicators. 
      The indicators are: pBlim, pBpa and pPrflim.
    \item[\texttt{vesselSum, vesselSumQ}:] Data frame with the indicators at vessel level. 
      The indicators are: catch, costs, discards, discRat, effort, fcosts, gva, income, landings, netProfit, 
      price, profits, quotaUpt, salaries, vcosts and profitability.
    \item[\texttt{vesselStkSum, vesselStkSumQ}:] Data frame with the indicators at vessel and stock level. 
      The indicators are: landings, discards, catch, price, quotaUpt, tacshare, discRat and quota.

    \item[\texttt{ecoSum\_damara}:] ecoSum built in the framework of Damara project.
    
    \item[\texttt{vesselStkSum, vesselStkSumQ}:] Data frame with the indicators at vessel and stock level. 
      The indicators are: landings, discards, catch, price, quotaUpt, tacshare, discRat and quota.

  \end{description}
  
  The data frames can be of wide or long format. In long format all the indicators are in the same column. 
  There is one column, indicator, for the name of the indicator and a second one value for the numeric value of the indicator. 
  In the wide format each of the indicators correspond with one column in the data frame. 
  The long format it is recommendable to work with ggplot2 functions for example while the wide format it is more efficient for memory allocation and speed of computations.

  The quantile version of the summaries, fooQ, returns the quantiles of the indicators. 
  In the long format as many columns as elements in prob are created. The name of the columns are the elements in prob preceded by a q. 
  In the wide format for each of the indicators as many columns as elements in prob are created. The names of the colums are the elements in prob preceded by q\_name\_of\_the\_indicator.


\subsection{Plotting functions} 

  Plotting functions to summarise the results of the FLBEIA output.

  \begin{description}
    
    \item[\texttt{plotFLBiols}:] For each stock, returns a pdf with plots using \texttt{FLBiols} object. With plots on biomass in numbers at age, mean weight at age, fecundity, natural mortality, maturity, spawning, recruitment and spawning stock biomass.
    \item[\texttt{plotFLFleets}:] For each fleet, returns a pdf with plots using \texttt{FLFleets} object. With plots on catch, discards, landings, capacity, crewshare, effort, fcost, effshare, and for each metier of this fleet: landings and discards at age in numbers and mean weight, alpha, beta and catch.q.
    \item[\texttt{plotCatchFl}:] Returns a pdf with plots using \texttt{FLFleets} and \texttt{advice} objects. Whith plots on landings, discards and price by fleet.
    \item[\texttt{plotEco}:] For each fleet, returns a pdf with plots using \texttt{FLFleets} object. With plots on capacity, costs, effort, profits by fleet.
  
  \end{description}
  
  
\subsection{Functions for joining iterations}

  \texttt{joinIter} function allows to join iterations of an \texttt{FLBEIA} output object with one unique iteration.
  This is very usefull in case that the different iterations are run separately in a cluster and we want to join them afterwards.
  
  The call to \texttt{joinIter} function within \texttt{FLBEIA} is done as:
	
	\begin{center}
		\texttt{joinIter(object, files, directory, Niters, elements, advice.ext = , fleets.ctrl.ext = } 
	\end{center}

  \noindent Arguments description:
	\begin{description}
		\item[\texttt{object}:] Character. Corresponds to the name of the object that must be 'joined'. 
		  This object must be the output of BEIA function.
		\item[\texttt{files}:] Character vector with the names of the files.
		  It has to be noted that each of the files must contain a single object.
		\item[\texttt{directory}:] The directory where the files are stored. The default is the current directory.
		\item[\texttt{Niters}:] Numeric vector with the number of iterations per object. 
		  If $length = 1$, then it is assumed that all objects have the same number of iterations.
		\item[\texttt{elements}:] The elements of the objects that must be joined. The default is to join all the elements.
		\item[\texttt{advice.ext}:] The type of advice that is given. Options are: \texttt{'TAC'} (default value) or \texttt{'TAE'}.
		\item[\texttt{fleets.ctrl.ext}:] Character. Name of the object in \texttt{fleets.ctrl} that needs iterations to be joined.
		  Default is \texttt{'seasonal.share'}.
	\end{description}

  Another alternative to join iterations is calculating summary statistics for each iteration and afterwards joining the summary results.

