
\subsection{Fourth level functions}

  These functions are called by the third level functions and, for the time being, are the functions 
in the lowest level within \texttt{FLBEIA}. 

\subsubsection{Stock-recruitment relationships} \label{sec:SRR}
	Stock-recruitment relationships are used, for example, within \texttt{ASPG} and \texttt{annualTAC} functions. The stock-recruitment 
	relationship used in \texttt{ASPG} is defined in the slot \texttt{model} of \texttt{FLSRsim} object and it defines the true
	recruitment dynamics of the stocks. Within \texttt{annualTAC}, the stock-recruitment relationship used is 
	defined in: 
		
		\begin{center}
			\texttt{advice.ctrl[[stock.name]][['sr']][['model']]}
		\end{center}

\noindent	element and it describes the 'observed' stock-recruitment dynamics (used) in the management process.

	In \texttt{FLCore} package there are several stock-recruitment relationships already defined and all can be 
	used within \texttt{FLBEIA}.
	Some of the functions available are:
	
	\begin{description}

		\item[\texttt{geomean}:] Recruitment is independent of the stock and equal to the geometric mean
	      of historical period.
	      		      $$ R = \alpha = \sqrt[n]{R_1\cdot\ldots\cdot R_n}$$

		\item[\texttt{bevholt}:] Beverton and Holt model with the following parameterization:
						$$ R = \frac{\alpha\cdot SSB}{(\beta + SSB)}$$
          where $\alpha$ is the maximum recruitment (asymptotically) and $\beta$ is
          the stock level needed to produce the half of maximum
          recruitment $\alpha /2$ $(\alpha, \beta >0)$.

		\item[\texttt{ricker}:] Ricker stock-recruitment model fit with the following parameterization:

                                $$R = \alpha\cdot SSB\cdot e^{-\beta\cdot SSB}$$

          where $\alpha$ is related to productivity and $\beta$ to density dependence.
          $\alpha$  is the recruit per stock unit at small stock levels. ($\alpha, \beta > 0$).

		\item[\texttt{segreg}:] Segmented regression stock-recruitment model fit:

                           \[
 							R = \begin{cases}
   									\alpha\cdot SSB       & \text{, if } SSB < \beta \\
   									\alpha\cdot \beta     & \text{, if } SSB \geq \beta
  								\end{cases}
							\]

          where $\alpha$ is the slope of the recruitment for stock levels below
          $\beta$ and $\alpha\cdot \beta $ is the mean recruitment for stock levels above
          $\beta$ ($\alpha, \beta > 0$).

		\item[\texttt{shepherd}:] Shepherd stock-recruitment model fit:

                              $$ R = \alpha \cdot \frac{SSB}{(1 + (S/\beta)^\gamma)}$$

          This model generalizes Beverton and Holt and Ricker models ($\gamma = 1$ corresponds with
          Beverton and Holt model, $\gamma > 1$ takes a Ricker-like shape and with $\gamma < 1$
          the curve rises indefinitely).

    \item[\texttt{bevholtAR1, rickerAR1, segregAR1}:] Beverton and Holt, Ricker and Segmented regression
      stock-recruitment models with autoregressive normal log
      residuals of first order. In the model fit the corresponding
      stock-recruitment model is combined with an autoregressive
      normal log likelihood of first order for the residuals. If
      $R_t$ is the observed recruitment and $\hat{R}_t$ is the predicted
      recruitment, an autoregressive model of first order is fitted
      to the log-residuals, $x_t = log(R_t/\hat{R}_t)$.

                           $$x_t = \rho\cdot x_{t-1} + \varepsilon$$

      where $\varepsilon \sim N(0,\sigma_{ar}^2)$.

    \item[\texttt{cushing}:] Cushing stock recruitment model fit:
              $$R = \alpha \cdot SSB ^ \beta$$
      where $\alpha, \beta > 0$.

    \item[\texttt{bevholtSV, rickerSV, segregSV, shepherdSV, cushingSV}:] Beverton and Holt, Ricker,
      Segemented regression, Shepherd and Cushing
      stock-recruitment models with $\alpha$ and $\beta$ parameterisation converted into
      steepness and virgin biomass ($s$ and $v$).

		\item[\texttt{rickerCa}:] Ricker stock-recruitment model with covariates, parameterised as:

          $$R = \alpha\cdot (1-\gamma \cdot covar) \cdot SSB \cdot e^{-\beta\cdot SSB}$$

  \end{description}
        
	Additionally, the following stock-recruitment relationships are defined in \texttt{FLBEIA} package:

	\begin{description}

		\item[\texttt{hockstick}:] Hockey stick stock-recruitment model fit:

              \[
 							R = \begin{cases}
   									\alpha\cdot S         & \text{, if } SSB < \beta \\
   									\alpha\cdot \beta     & \text{, if } SSB \geq \beta
  								\end{cases}
							\]

          where $\alpha$ is the slope of the recruitment for stock levels below
          $\beta$ and $\alpha\cdot \beta $ is the mean recruitment for stock levels above
          $\beta$ ($\alpha, \beta > 0$).

		\item[\texttt{redfishRecModel}:] Redfish recruitment model developed by Benjamin Planque, with the formula:

          \[
          R_y = redfishRec(R_{y-1},\sigma, minrec, maxrec) \cdot
            \begin{cases}
              \frac {SSB} \alpha  & \text{, if } SSB < \alpha \\
              1                   & \text{, if } SSB \geq \alpha
            \end{cases}
          \]
          Being:
            $$redfishRec(R_{y-1},\sigma, minrec, maxrec) = R_{y-1} + rnorm(n=1,mean=0,sd=\sigma)$$
            $$minrec \leq redfishRec(R_{y-1},\sigma, minrec, maxrec) \geq maxrec$$
          Where $R_{y-1}$ corresponds to previous year's recruitment, $\gamma$ the standard deviation of
          the historic recruitment and \texttt{minR}, \texttt{maxR}
          to the minimun and maximum recruitments historically observed, respectively.

    \item[\texttt{aneRec_pil}, \texttt{pilRec_ane}:] Ricker models with a covariate in the exponent, for simulating
      stk1 recruitment assuming predation on its eggs by stk2, given by the formula:

        $$R_{stk1} = \alpha\cdot SSB_{stk1} \cdot e^{-\beta \cdot SSB_{stk1} + \gamma \cdot SSB_{stk2}}$$

    \item[\texttt{ctRec}:] constant recruiment. There is not recruitment modelling and therefore expected
        recruitment values for the projection period has to be fixed a priori.

  \end{description}

  There could be more stock-recruitment relationships defined in \texttt{FLCore} or \texttt{FLBEIA}, thus, 
  if you are interested in using 
        a model not defined here take a look at \texttt{SRModels} help page in \texttt{FLCore} package. 
        New stock-recruitment models to be used in \texttt{FLSRsim} class can be defined in two ways:
               
		\begin{enumerate}
			\item Using a formula in slot \texttt{model}:
					$$ rec \sim \Phi(X)$$
				where $\Phi$ is a function of \texttt{ssb} and parameters and covariates stored in \texttt{params} and  \texttt{covar} slots
				respectively.
			\item Defining a function in \texttt{R}, \texttt{foo <- function(X)}, and using the name of the function, \texttt{foo}, 
					in slot \texttt{model}. The function arguments must be among \texttt{ssb} and parameters and covariates stored in
					\texttt{params} and  \texttt{covar} slots respectively.
					 
		\end{enumerate}
        

\subsubsection{Catch production functions} \label{sec:CprodFun}

	The catch production functions can be different for the same third level effort model.
	Currently, there are three catch production functions available. 
	The first two correspond with Cobb-Douglas production functions \citep{Clark1990, Cobb1928}
	but in one case the model operates at stock level and in the second one at age class level.
	The last one is used when information on effort is lacking and consequently catches are
	set independently from effort.
 	
 	\paragraph{\texttt{CobbDouglasBio}: Cobb-Douglas production function at stock level} \hspace{0pt} \smallskip
 	
 	The total catch of the fleet is calculated according to the Cobb-Douglas production function:
		
\begin{equation}  \label{eq:Cobb_Doug}
	C = q\cdot E^{\alpha} \cdot B^{\beta}
\end{equation}

  \noindent	where $C$ denotes total catch and $B$ total biomass (both in weight), $q$ the catchability and $E$ the effort. 
	$\alpha$ and $\beta$ are the elasticity parameters associated to labor and capital 
	(biomass in this case), respectively. These parameters are associated
	to the existing technology. 

	As $\alpha$ and $\beta$ parameters depend on the stock and the technology, Cobb-Douglas function is
	applied at metier level. Thus, the catch of a certain fleet $f$ is given by:

\begin{equation}  \label{eq:Cobb_Doug_fleet}
	C_f = \sum_{m \in M_f} q_{f,m}\cdot B^{\beta _{f,m}} \cdot (E_f\cdot \delta_{f,m})^{\alpha_{f,m}}
\end{equation}
 
	\noindent where $M_f$ represents the set of metiers of fleet $f$ and $\delta$ the effort share among metiers.


	\subparagraph{Derivation of Catch-at-age.}
	%~~~~~~~~~~~~~~~~~~~~~~~~~~~~~~~~~~~~~~~~
	Once the total catch is calculated, it is divided into catch at age using selectivity at age, $s_{a,f,m}$, and
	biomass at age in the population, $B_{a}$:
		
	\begin{equation}  \label{eq:catch_at_age}
		C_{a,f,m} = \frac{C_{f,m}}{\sum_a s_{a,f,m}\cdot B_a} \cdot s_{a,f,m}\cdot B_a
	\end{equation}
	
	Derivation of Equation~\ref{eq:catch_at_age}:
			
	\begin{itemize}
		\item If the whole population were accessible to the gear, the catch of age $a$ would be:
			$$s_{a,f,m}\cdot B_a$$
		\item Thus, if the whole population were accessible to the gear, the total catch we could obtain would be:
			$$\sum_a s_{a,f,m}\cdot B_a$$ 
		\item But, the actual total catch is $C_f$, so theoretically the proportion of the population that have been
		  accessible is\footnote{If all the age classes were not accessible or completely accessible
		 we would replace $s_{a,f,m}$ by $\acute{s_{a,f,m}} = \gamma_{a,f,m} \cdot s_{a,f,m}$ where $\gamma_{a,f,m}$ 
		 is the proportion of individuals of age $a$ accessible to metier $m$ in fleet $f$.}:
			$$\frac{C_{f,m}}{\sum_a s_{a,f,m}\cdot B_a}$$ 

		\item Then, if we assume the population is homogeneously distributed we arrive to Equation~\ref{eq:catch_at_age}.
	\end{itemize}
		
	The catch at age is then further disaggregated in landings- and discards-at-age using landings' and discards' specific selectivity:
		
	\begin{equation}  \label{eq:land_disc}
		L_{a,f,m} = \frac{sl_{a,f,m}}{s_{a,f,m}}\cdot C_{a,f,m} \quad  \text{and} \quad  D_{a,f,m} = \frac{sd_{a,f,m}}{s_{a,f,m}}\cdot C_{a,f,m}
	\end{equation}

 	
 	\paragraph{\texttt{CobbDouglasAge}: Cobb-Douglas production function at age-class level} \hspace{0pt} \smallskip

 	 	The  catch of the fleets is calculated according to the Cobb-Douglas production function 
		applied at age-class level, i.e.:
		
\begin{equation}  \label{eq:Cobb_Doug_age}
	C = \sum_a C_a = q_a\cdot E^{\alpha_a} \cdot B_a^{\beta_a}
\end{equation}

	\noindent where $C$ denotes catch and $B$  biomass (both in weight), $q$ the catchability, $E$ the effort
	and $a$ the subscript for age. $\alpha$ and $\beta$ are the elasticity parameters associated to labor and capital 
	(biomass in this case), respectively. These parameters are associated
	to the existing technology. 

	As $\alpha$ and $\beta$ parameters dependent on age classes and technology, Cobb-Douglas function is
	applied at metier level. Thus, the catch of a certain fleet $f$ is given by:

\begin{equation}  \label{eq:Cobb_Doug_fleet_age}
	C_f = \sum_a C_{a,f} = \sum_{m \in M_f} \sum_a q_{a,f,m}\cdot B_a^{\beta _{a,f,m}} \cdot (E_f\cdot \delta_{f,m})^{\alpha_{a,f,m}}
\end{equation}
 
	\noindent where $M_f$ represents the set of metiers of fleet $f$, $\delta$ the effort share among metiers and $m$ 
	is the subscript that indicates the metier.

 	\paragraph{\texttt{seasonShare}: Catches estimation given season share allocation by metier} \hspace{0pt} \smallskip

 	 	In case that there is no information on the effort, and therefore effort is not limiting the catches, the catch at age of
 	 	each metier can be calculated according to a previously defined season share allocation by metier. 
 	 	This function is only valid for metiers which target only one stock.
 	 	Alternatively, the seasonal share for one stock can be set equal to the one of a reference fleet (usually one which has 
 	 	information on effort).
 	 	
 	 	Two arguments need to be declared as elements of \texttt{fleets.ctrl} if this function is used:
 	 	\begin{description}
			\item[\texttt{effort.model}:] 'fixedEffort'. 
			This argument must be declared at fleet level (i.e. \\ \texttt{fleets.ctrl[[fleet.name]]\$effort.model})
			\item[\texttt{catch.model}:] 
			Argument is used to specify the catch production function that will be used to generate the catch 
			and it must be declared at fleet and stock level (i.e. \\ \texttt{fleets.ctrl[[fleet.name]][[stock.name]]\$catch.model}). 
			That catch production model corresponds with a fourth level function (see Section~\ref{sec:CprodFun}).
			\item[\texttt{catch.dependence}:] This argument needs to be set only 
			  when aiming to set the seasonal share of one stock equal to the same stock in other
			  reference fleet and it must be declared at fleet and stock level 
			  (i.e. \texttt{fleets.ctrl[[fleet.name]][[stock.name]]\$catch.dependence}).
			  Value has to be set equal to the name of this reference fleet.
 	 	\end{description}
 	 	
% 		Two arguments need to be declared as elements of \texttt{fleets.ctrl} if this function is used, \texttt{effort.model = 'fixedEffort'} 
% 	and \texttt{catch.model}. The last argument is used to specify the catch production function that will be used to generate the catch. 
%   Note that first argument must be declared at fleet level (i.e \texttt{fleets.ctrl[[fleet.name]]\$effort.model}), 
%   second argument at fleet and stock level (i.e. \texttt{fleets.ctrl[[fleet.name]][[stock.name]]\$catch.model})
% 	and that catch production model corresponds with a fourth level function.
% 	When aiming to set the seasonal share of one stock equal to the same stock in other reference fleet, 
% 	then an additional argument, \texttt{catch.dependence}, needs to be set at fleet and stock level 
% 	(i.e. \texttt{fleets.ctrl[[fleet.name]][[stock.name]]\$catch.dependence}) equal to the name of this reference fleet.
	
	
\subsubsection{Costs functions}

	Cost functions have been developed in order to be used within \texttt{fleets.om}. As cost structure could 
	differ among fleets it has been defined as fourth level function and it works at fleet level. 
	In principle, it could be useful in both tactic and strategic dynamics of fleets. 
	 
	\paragraph{\texttt{TotalCostsPower}: Total costs power function} \hspace{0pt} \smallskip

	This function sums up the fixed costs ($FxC$) and the power functions of cost per unit of effort ($CostPUE$),
	crew share per unit of landings ($CSPUL$) and capital cost per unit capital ($CapCostPUC$), mathematically:
	
	\begin{align}
		Cost_f  = & FxC_f + \sum_{m} (CostPUE_{f,m} \cdot E_f \cdot \tau_{f,m})^{\gamma_{1 f,m}} + 
		 \nonumber \\
			 & \sum_{st} \sum_{m} CSPUL_{f,m,st} \cdot L_{f,m,st}^{\gamma_{2 f,m,st}} +  CapCostPUC_f\cdot Cap_f^{\gamma_{3 f}}
	\end{align}
	
	The fixed cost are given at fleet level, $f$, cost per unit of effort at metier level, $m$, and crew share at 
	fleet, metier and stock, $st$, level. $\gamma_{1 f,m}$ is the exponent of effort  at fleet and metier level 
	in cost of effort addend, $\gamma_{2 f,m,st}$ the exponent of landing at fleet, metier and stock level in crew share  
	cost addend and $\gamma_{3 f}$ is the exponent of capital at fleet level in capital cost addend.  



