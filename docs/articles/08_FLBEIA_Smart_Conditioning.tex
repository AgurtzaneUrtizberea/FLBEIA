
\section{Smart conditioning} \label{sec:FLBEIASmartCond}

\subsection{Functions}  \label{sec:SmartCondFun}

	\texttt{FLBEIA} requires a number of input arguments to run a simulation; such as \texttt{biols}, \texttt{SRs}, \texttt{BDs}, \texttt{fleets}, \texttt{covars}, \texttt{indices}, \texttt{advice}, \texttt{main.ctrl}, \texttt{biols.ctrl}, \texttt{fleets.ctrl}, \texttt{covars.ctrl}, \texttt{obs.ctrl}, \texttt{assess.ctrl} and \texttt{advice.ctrl}. These objects contain biological and economical historical and projection data, and also point the functions that are going to be used by the model. Here we introduce and explain the functions that have been generated to facilitate the creation of these objects:

\begin{description}
  \item[\texttt{create.biols.data}:] It generates an \texttt{FLBiol} object for each stock, and includes all of them in a \texttt{FLBiols} object. It returns an object that could be used as \texttt{biols} argument in \texttt{FLBEIA} function. The function requires historical data of weight, abundance, natural mortality, fecundity and spawning. In the projection years, natural mortality, fecundity and spawning, are assumed equal to the average of the historical years that the user specifies in \texttt{stk\_biol.proj.avg.yrs}. The descriptions and format of the arguments required by the function are presented in Table~\ref{tb:A4.table1}.
\end{description}

\begin{description}
  \item[\texttt{create.SRs.data}:] It generates a list with  \texttt{FLSRsim} objects and returns an object that could be used as \texttt{SRs} argument in \texttt{FLBEIA} function. This function does not calculate stock-recruitment function's parameter values; therefore, they must be calculated previously by the user. In case that the proportion of spawning per season is not defined in the projection years, then it is assumed equal to the average of the historical years range that the user defines. If uncertainty is not an input, then there is not uncertainty. The descriptions and format of the arguments required by the function are presented in Table~\ref{tb:A4.table2}.
\end{description}

\begin{description}
  \item[\texttt{create.BDs.data}:] It generates a list with \texttt{FLBDsim} objects. It returns an object that could be used as \texttt{BDs} argument in \texttt{FLBEIA} function. This function does not calculate biomass dynamics function's parameter values, so they must be introduced by the user. If uncertainty is not an input, then the function assumes no uncertainty. The descriptions and format of the arguments required by the function are presented in Table~\ref{tb:A4.table3}.
\end{description}

\begin{description}
  \item[\texttt{create.fleets.data}:] It generates an \texttt{FLFleet} object for each fleet and includes all of them in an \texttt{FLFleets} object. It returns an object that could be used as \texttt{fleets} argument in \texttt{FLBEIA} function. The function requires historical data of effort per fleet, effort share between m\'etiers and, landings and weight at age per stock. Notice that the input data of landings\' weight has the same name as stock weight input in \texttt{create.biols.data}. In case that discards data are available, then its weight is assumed the same as for landings. The descriptions and format of the arguments required by the function are presented in Table~\ref{tb:A4.table4}.
  The function assumes that when historical data of effort, fixed cost, capacity or crewshare are introduced, then the projection values of each of them are the average of the years that the user sets in \texttt{fl.proj.avg.yrs}; in the case of effort share and variable cost per metier, is set in \texttt{fl.met\_proj.avg.yrs}; and in the case of landings at age, discards at age and price, in \texttt{fl.met.stk\_proj.avg.yrs}. If the Cobb-Douglas parameters, \texttt{alpha}, \texttt{beta} and \texttt{q} (see more information on Section~\ref{sec:CprodFun}), are not introduced as inputs, then they are created by the \texttt{calculate.CBparam} function, where it assumes that \texttt{alpha} and \texttt{beta} are equal to 1 and \texttt{q} is the ratio between total catch and effort per metier multiplied by the stock abundance.
\end{description}

\begin{description}
  \item[\texttt{create.indices.data}:] It generates a list with all the stocks and for each stock a list with \texttt{FLIndex} objects. It returns an object that could be used as \texttt{indices} argument in \texttt{FLBEIA} function. The descriptions and format of the arguments required by the function are presented in Table~\ref{tb:A4.table5}.
\end{description}

\begin{description}
  \item[\texttt{create.advice.data}:] It generates a list with the advice for each of the stocks. It returns an object that could be used as \texttt{advice} argument in \texttt{FLBEIA} function. In case that the values of TAC and TAE are not introduced in the projection years, then the model assumes that they are equal to the average of the historical data that the user defines in \texttt{stk\_advice.avg.yrs}. When quota share is not defined in the projection years, then the function calculates it for each stock as the ratio between catch per fleet and total catch. The descriptions and format of the arguments required by the function are presented in Table~\ref{tb:A4.table6}.
\end{description}

\begin{description}
  \item[\texttt{create.biols.ctrl}:] It creates an object with the name of the growth function for each stock. The object that returns this function can be used as \texttt{biols.ctrl} argument in \texttt{FLBEIA} function.
\end{description}

\begin{description}
  \item[\texttt{create.fleets.ctrl}:] It creates an object with the fleet dynamics function that is applied for each fleet. The object that returns this function can be used as \texttt{fleets.ctrl} argument in \texttt{FLBEIA} function.
\end{description}

\begin{description}
  \item[\texttt{create.covars.ctrl}:] It creates an object with the function that is applied to the covariate. The object that returns this function can be used as \texttt{covars.ctrl} argument in \texttt{FLBEIA} function.
\end{description}

\begin{description}
  \item[\texttt{create.obs.ctrl}:] It creates a function with the observed function for each stock. The object that returns this function can be used as \texttt{obs.ctrl} argument in \texttt{FLBEIA} function.
\end{description}

\begin{description}
  \item[\texttt{create.advice.ctrl}:] I creates an object with the harvest control rule for each stock, and its parameter values. The object that returns this function can be used as \texttt{advice.ctrl} argument in \texttt{FLBEIA} function.
\end{description}

\begin{description}
  \item[\texttt{create.assess.ctrl}:] It creates an object with the name of the kind of assessment that is applied to each stock. The object that returns this function can be used as \texttt{assess.ctrl} argument in \texttt{FLBEIA} function.
\end{description}

We generate other functions to simplify the previous ones:

\begin{description}
  \item[\texttt{create.list.stks.flqa}:] It creates a list of \texttt{FLQuant} objects for each stock with the corresponding dimensions and dimension names. One of the dimensions in the \texttt{FLQuant} is age; in a range between the minimum and maximum age. The descriptions and format of the arguments required by the function are presented in Table~\ref{tb:A4.table7}.
\end{description}

\begin{description}
  \item[\texttt{create.list.stks.flq}:] It creates a list of \texttt{FLQuant} objects for each stock with the corresponding dimensions and dimension names. One of the \texttt{FLQuant} is age, defined as \texttt{all}. The descriptions and format of the arguments required by the function are presented in Table~\ref{tb:A4.table8}.
\end{description}

\begin{description}
  \item[\texttt{calculate.CBparam}:] It creates a list with the three Cobb-Douglas parameters: \texttt{alpha}, \texttt{beta} and \texttt{q}. It assumes that \texttt{alpha} and \texttt{beta} are equal to 1, and \texttt{q} is calculated as the ratio between total catch and the multiplication of effort per metier and stock abundance. The descriptions and format of the arguments required by the function presented in Table~\ref{tb:A4.table9}.
\end{description}

\subsection{Examples}  \label{sec:SmartCondEx}

Each test case has three folders: \texttt{data}, \texttt{R} and \texttt{plots}. In the folder called \texttt{data} are the input data in \texttt{csv} format. In the folder called \texttt{R} is the conditioning script, which: (i) creates the input objects for \texttt{FLBEIA} and saves them in the same folder with \texttt{RData} format; and (ii) runs the model and makes plots that are saved in the folder \texttt{plots}. In \texttt{R} folder, there is another folder called \texttt{results} with the output of the \texttt{FLBEIA} model in \texttt{RData} format.

\begin{description}
  \item[1 stock, 1 fleet and 1 season:] This test case analyzes the dynamics of a fictitious stock (\texttt{SBR}) and a fictitious fleet (\texttt{DLL}). The fleet has one metier, with the same name as the fleet (\texttt{DLL}). This case study is an example on how to include in the model: an assessment (XSA), observations data (disaggregated in ages) or elastic price. 

The biological historical data available in this study range from 1990 to 2009 and \texttt{FLBEIA} is run from 1990 to 2025, with first year of projection in 2010. Biological data are described by stock abundance, weight, spawning, fecundity and natural mortality with three iterations, but only abundance and weight have some variability. Biological data are age specific in a range of ages between 1 and 12 years and with only one season.  The minimum year to calculate the average \texttt{f} is set as 1 and the maximum as 12. The values of weight, spawning, fecundity and natural mortality in the projection period are assumed to be equal to the average between 2007 and 2009.

The fleet has historical information on effort, crewshare, capacity and fixed costs. There is only one metier, so effort share is one. Landings and discards, as well as Cobb-Douglas parameters (\texttt{alpha},\texttt{beta},\texttt{q}) values are introduced as input.

The model applies an age-structured population growth model and catch model. Beverton-holt autoregressive model is assumed as stock-recruitment relationship and uncertainty is not introduced in the projection. 
\end{description}

\begin{description}
    \item[2 stocks, 2 fleets and 4 seasons:] This test case analyzes the dynamics of two fictitious stocks (\texttt{stk1} and \texttt{stk2}), with two fictitious fleets (\texttt{fl1} and \texttt{fl2}). Each of the fleets has two metiers (\texttt{met1} and \texttt{met2}) and all the metiers catch both stocks. This case study simulates the management using an ICES harvest control rule for stock \texttt{stk1} and annual TAC for \texttt{stk2}. The effort function of both fleets is different; fixed effort is assumed for fleet \texttt{fl1} and simple mixed fisheries behavior, limited by the minimum catch of both stocks, for fleet \texttt{fl2}. There is no assessment in this study and perfect observation is assumed.

The biological historical data available for both stocks range from 1990 to 2008. The simulation time covers from 1990 to 2025, with 2009 as the first projection year. Biological data are described by stock abundance, weight, spawning, fecundity and natural mortality. Biological data of stock \texttt{stk1} are age structured in a range from 0 to 15 years, but not for stock \texttt{stk2}, which is modelled in biomass. The minimum age to calculate the average \texttt{f} for stock \texttt{stk1} is set as 1 and the maximum as 5. Stock \texttt{stk1} has 4 spawning seasons, while stock \texttt{stk2} only one, then we set different units for both. The values of weight, spawning, fecundity and natural mortality in the projection period are assumed to be equal to the average between 2006 and 2008 for both stocks.

Both fleets have historical information on effort and capacity, but only fleet \texttt{fl2} has fixed cost data. For each fleet and metier, effort share data are available and only for fleet \texttt{fl2} are variable costs. For each fleet and metier, there is age structured data for stock \texttt{stk1} and in biomass for \texttt{stk2}. Both fleets have landings and discards data for each stock, but in the case of stock \texttt{stk2} discards data of metier \texttt{met2} in fleet \texttt{fl1} are missing. Cobb-Douglas parameters (\texttt{alpha},\texttt{beta},\texttt{q}) are not introduced as input; therefore the function \texttt{create.fleets.data} calls \texttt{calculate.CBparam} function to calculate them.

Since the biological and economical historic data of stock \texttt{stk1} are age specific, then the model allows using age-structured population growth and catch model, while in the case of stock \texttt{stk2} they must be based on biomass. Beverton-Holt model is applied as stock-recruitment relationship for stock \texttt{stk1}, with 4 spawning seasons, and Pella-Tomlinson biomass dynamics model for stock \texttt{stk2}. In both cases the parameters in the projection period are introduced as input.  
\end{description}

