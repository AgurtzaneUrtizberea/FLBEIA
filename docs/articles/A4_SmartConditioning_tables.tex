%\appendix

\begin{landscape}

\section{Smart conditioning - function's arguments description} \label{sec:SmartCond_tab}

\setcounter{figure}{0} 
\setcounter{table}{0}

% \subsection{Tables}

%TABLE 1
\begin{table}[!ht]

  %\begin{center}
  \centering
  \begin{footnotesize}
    
    \caption{Description of the arguments of the function \texttt{create.biols.data}. 
      In the table we assume that \texttt{stk} is the name of the stock. All the arguments are required.}
    
    \label{tb:A4.table1}
    
    \begin{threeparttable}
    
      \begin{tabular}{lllll} %{c|c}
        \hline 
        Argument & & class & Dimension & Definition\\
        \hline
        yrs & & vector & 3 &	c( first.yr, proj.yr, last.yr)\\
          & first.yr & numeric & 1 & First year of simulation\\
          & proj.yr  & numeric & 1 & First year of projection\\
          & last.yr  & numeric & 1 & Last year of projection\\
        ni & & numeric &	1 &	Number of iterations\\
        ns & & numeric &	1 &	Number of seasons\\
        stks.data & &	list & number of stocks &	List with the name of the stocks and the following elements:\\
          & stk.unit    &	numeric &	1 &	Number of units\\
          & stk.age.min &	numeric &	1 &	Minimum age\\ 
          & stk.age.max &	numeric &	1 &	Maximum age\\
          & stk\_n.flq    &	FLQuant &	[na,ny(hist),1/nu(stock),1/ns,1/ni] &	Abundance in numbers at age\\
          & stk\_wt.flq   &	FLQuant &	[na,ny(hist),1/nu(stock),1/ns,1/ni] &	Weight at age\\
          & stk\_m.flq    &	FLQuant &	[na,ny(hist),1/nu(stock),1/ns,1/ni] &	Natural mortality mortality rate\\
          & stk\_fec.flq  &	FLQuant &	[na,ny(hist),1/nu(stock),1/ns,1/ni] &	Fecundity\\
          & stk\_mat.flq  &	FLQuant &	[na,ny(hist),1/nu(stock),1/ns,1/ni] &	Percentage of mature individuals\\
          & stk\_spwn.flq &	FLQuant &	[na,ny(hist),1/nu(stock),1/ns,1/ni] &	Proportion of time step at spawning\\
          & stk\_range.min       &	numeric &	1 &	Minimum age\\
          & stk\_range.max       &	numeric &	1 &	Maximum age\\
          & stk\_range.plusgroup &	numeric &	1 &	Plusgroup age\\
          & stk\_range.minyear   &	numeric &	1 &	Minimum year\\
          & stk\_range.maxyear   &	numeric &	1 &	Maximum year\\
          & stk\_range.minfbar   &	numeric &	1 &	Minimum age for calculating average fishing mortality\\
          & stk\_range.maxfbar   &	numeric &	1 &	Maximum age for calculating average fishing mortality \\
          & stk\_biol.proj.avg.yrs &	vector &	any &	Historic years to calculate averages (in spwn, fec, m and wt)\\
          &  &	&	&	for the projection period\\
        \hline
      \end{tabular}
      
      \begin{tablenotes}
        % \small
        \item na: number of age (from min.age to max.age)
        \item ny(hist): number of historic years (from first.yr to proj.yr-1)
        \item 1/nu(stock): 1 or number of units of the stock
        \item 1/ns: 1 or number of seasons
        \item 1/ni:  1 or number of iterations
      \end{tablenotes}
      
    \end{threeparttable}
  \end{footnotesize}
  %\end{center}

\end{table}			

	

%TABLE 2
\begin{table}[!ht]

  %\begin{center}
  \centering
  \begin{footnotesize}
    
    \caption{Description of the arguments of the function \texttt{create.SRs.data}. 
      In the table we assume that \texttt{stk} is the name of the stock. 
      The arguments with superscript \textsuperscript{*} are optional arguments.}
      
    \label{tb:A4.table2}
    
    \begin{threeparttable}
    
      \begin{tabular}{lllll} %{c|c}
        \hline 
        Argument & & class & Dimension & Definition\\
        \hline
        yrs & & vector & 3 &	c( first.yr, proj.yr, last.yr)\\
          & first.yr & numeric & 1 & First year of simulation\\
          & proj.yr  & numeric & 1 & First year of projection\\
          & last.yr  & numeric & 1 & Last year of projection\\
        ni & & numeric &	1 &	Number of iterations\\
        ns & & numeric &	1 &	Number of seasons\\
        stks.data & &	list & number of stocks &	List with the name of the stocks and the following elements:\\
          & stk.unit          &	numeric &	1 &	Number of units\\
          & stk.age           &	numeric &	1 &	Number of age classes\\
          & stk\_sr.model     & character &	1 &	Name of the SR model\\ 
          & stk\_params.n     &	vector &	1  &	Number of parameters\\ 
          & stk\_params.name  &	vector &	stk\_params.n &	Name of the parameters\\
          & stk\_params.array &	array &	[stk\_params.n,ny,ns,1/ni] &	Parameter values\\
          & stk\_rec.flq  &	FLQuant &	[1,ny(hist),1/nu(stock),1/ns,1/ni] &	Recruitment values\\
          & stk\_ssb.flq  &	FLQuant &	[1,ny(hist),1/nu(stock),1/ns,1/ni] &	Spawning stock values\\
          & stk\_uncertainty.flq\textsuperscript{*} & FLQuant & [1,ny,1/nu(stock),1/ns,1/ni] & Uncertainty\\
          & stk\_proportion.flq  &	FLQuant &	[1,ny,1/nu(stock),1/ns,1/ni] &	Recruitment distribution in each time step. For details see \texttt{FLSRsim}\\
          & stk\_prop.avg.yrs &	vector & any & Historical years to calculate the proportion average\\
          & stk\_timelag.matrix  &	matrix &	(2,ns) &	Timelag between the spawning an recruitment (time.lag.yr, time.lag.ns)\\
          & & & & For details see \texttt{FLSRsim}\\
          & stk\_range.min       &	numeric &	1 &	Minimum age\\
          & stk\_range.max       &	numeric &	1 &	Maximum age\\
          & stk\_range.plusgroup &	numeric &	1 &	Plusgroup age\\
          & stk\_range.minyear   &	numeric &	1 &	Minimum year\\
          % & stk\_range.maxyear   &	numeric &	1 &	Maximum year\\
        \hline
      \end{tabular}
      
      \begin{tablenotes}
        % \small
        \item na: number of age (from min.age to max.age)
        \item ny(hist): number of historic years (from first.yr to proj.yr-1)
        \item ny: number of years (from first.yr to last.yr)
        \item 1/nu(stock): 1 or number of units of the stock
        \item 1/ns: 1 or number of seasons
        \item 1/ni: 1 or number of iterations
        \item ns: number of seasons
      \end{tablenotes}
      
    \end{threeparttable}
  \end{footnotesize}
  %\end{center}

\end{table}			



%TABLE 3
\begin{table}[!ht]

  %\begin{center}
  \centering
  \begin{footnotesize}
    
    \caption{Description of the arguments of the function \texttt{create.BDs.data}. 
      In the table we assume that \texttt{stk} is the name of the stock. 
      The arguments with \textsuperscript{*} are optional arguments.}
    
    \label{tb:A4.table3}
    
    \begin{threeparttable}
    
      \begin{tabular}{lllll} %{c|c}
        \hline 
        Argument & & class & Dimension & Definition\\
        \hline
        yrs & & vector & 3 &	c( first.yr, proj.yr, last.yr)\\
          & first.yr & numeric & 1 & First year of simulation\\
          & proj.yr  & numeric & 1 & First year of projection\\
          & last.yr  & numeric & 1 & Last year of projection\\
        ni & & numeric &	1 &	Number of iterations\\
        ns & & numeric &	1 &	Number of seasons\\
        stks.data & &	list & number of stocks &	List with the name of the stocks and the following elements:\\
          & stk.unit          &	numeric &	1 &	Number of units\\
          & stk\_bd.model     & character &	1 &	Name of the BD model\\ 
          & stk\_params.name  &	vector &	np &	Name of the parameters\\ 
          & stk\_params.array &	vector &	np &	Parameter values\\
          & stk\_biomass.flq  &	FLQuant &	[1,ny(hist),1/nu(stock),1/ns,1/ni] &	Biomass values\\
          & stk\_catch.flq    &	FLQuant &	[1,ny(hist),1/nu(stock),1/ns,1/ni] &	Catch values\\
          & stk\_range.min       &	numeric &	1 &	Minimum age\\
          & stk\_range.max       &	numeric &	1 &	Maximum age\\
          & stk\_range.plusgroup &	numeric &	1 &	Plusgroup age\\
          & stk\_range.minyear   &	numeric &	1 &	Minimum year\\
          % & stk\_range.maxyear   &	numeric &	1 &	Maximum year\\
          & stk\_alpha        &	numeric &	1 &	Maximum variability of carrying capacity\\
          & stk\_gB.flq\textsuperscript{*} & FLQuant & [1,ny(hist),1/nu(stock),1/ns,1/ni] & Surplus production values\\
          & stk\_uncertainty.flq\textsuperscript{*} & FLQuant & [1,ny,1/nu(stock),1/ns,1/ni] & Uncertainty\\
        \hline
      \end{tabular}
      
      \begin{tablenotes}
        % \small
        \item na: number of age (from min.age to max.age)
        \item ny(hist): number of historic years (from first.yr to proj.yr-1)
        \item ny: number of years (from first.yr to last.yr)
        \item 1/nu(stock): 1 or number of units of the stock
        \item 1/ns: 1 or number of seasons
        \item 1/ni:  1 or number of iterations
        \item np: number of parameters in BD model
      \end{tablenotes}
      
    \end{threeparttable}
  \end{footnotesize}
  %\end{center}

\end{table}	


	
%TABLE 4
\begin{table}[!ht]

  %\begin{center}
  \centering
  \begin{footnotesize}
    
    \caption{Description of the arguments of the function \texttt{create.fleets.data}. 
      In the table we assume that \texttt{stk} is the name of the stock,\texttt{fl} the name of the fleet and \texttt{met} the name of the metier. 
      The arguments with \textsuperscript{*} are optional arguments.}
    
    \label{tb:A4.table4}
    
    \begin{threeparttable}
    
      \begin{tabular}{lllll} %{c|c}
        \hline 
        Argument & & class & Dimension & Definition\\
        \hline
        yrs & & vector & 3 &	c( first.yr, proj.yr, last.yr)\\
          & first.yr & numeric & 1 & First year of simulation\\
          & proj.yr  & numeric & 1 & First year of projection\\
          & last.yr  & numeric & 1 & Last year of projection\\
        ni & & numeric &	1 &	Number of iterations\\
        ns & & numeric &	1 &	Number of seasons\\
        fls.data & &	list & number of fleets &	List with the name of the fleets and the following elements:\\
          & fl.met      & vector &	number of metiers in 'fl' &	Name of the metiers in the fleet 'fl'\\
          & fl.met.stks & vector &	number of stocks in 'fl.met' &	Name of the stocks in the metier 'met' and fleet 'fl'\\
          & fl\_effort.flq            & FLQuant & [na,ny(hist),1/nu(stock),1/ns,1/ni] &	Effort for 'fl' fleet\\
          & fl\_capacity.flq\textsuperscript{*}  & FLQuant & [na,ny(hist),1/nu(stock),1/ns,1/ni] & Capacity of 'fl' fleet\\
          & fl\_fcost.flq\textsuperscript{*}     & FLQuant & [na,ny(hist),1/nu(stock),1/ns,1/ni] & Fixed costs for 'fl' fleet\\
          & fl\_crewshare.flq\textsuperscript{*} & FLQuant & [na,ny(hist),1/nu(stock),1/ns,1/ni] & Crewshare for 'fl' fleet\\
          & fl.met\_effshare.flq      & FLQuant & [na,ny(hist),1/nu(stock),1/ns,1/ni] &	Effort share for fl' fleet and 'met' metier\\
          & fl.met\_vcost.flq\textsuperscript{*} & FLQuant & [na,ny(hist),1/nu(stock),1/ns,1/ni] & Variable costs for 'fl' fleet and 'met' metier\\
          & fl.met.stk\_landings.n.flq &	FLQuant & [na,ny(hist),1/nu(stock),1/ns,1/ni] &	Landings in numbers at age for fl' fleet,'met' metier and 'stk' stock\\
          & fl.met.stk\_landings.wt.flq\textsuperscript{*} & FLQuant & [na,ny(hist),1/nu(stock),1/ns,1/ni] & Mean weight of landings at age for 'fl' fleet and 'met' metier\\
          & fl.met.stk\_discards.n.flq\textsuperscript{*} & FLQuant & [na,ny(hist),1/nu(stock),1/ns,1/ni] &	Discards in numbers at age for 'fl' fleet and 'met' metier\\        
          & fl.met.stk\_discards.wt.flq\textsuperscript{*} & FLQuant & [na,ny(hist),1/nu(stock),1/ns,1/ni] & Mean weight at age in discards for 'fl' fleet and 'met' metier\\
          & fl.met.stk\_price.flq\textsuperscript{*}   & FLQuant & [na,ny(hist),1/nu(stock),1/ns,1/ni] & Price at age for 'stk' stock in 'fl' fleet and 'met' metier\\
          & fl.met.stk\_alpha.flq\textsuperscript{*}   & FLQuant & [na,ny,1/nu(stock),1/ns,1/ni] & Cobb-Douglas alpha parameter for 'fl' fleet, 'met' metier and 'stk' stock\\
          & fl.met.stk\_beta.flq\textsuperscript{*}    & FLQuant & [na,ny,1/nu(stock),1/ns,1/ni] & Cobb-Douglas beta parameter for 'fl' fleet, 'met' metier and 'stk' stock\\
          & fl.met.stk\_catch.q.flq\textsuperscript{*} & FLQuant & [na,ny,1/nu(stock),1/ns,1/ni] & Cobb-Douglas catch.q parameter for 'fl' fleet, 'met' metier and 'stk' stock\\
          & fl\_proj.avg.yrs &	vector &	any &	Historic years to calculate averages (in effort, fcost, crewshare, and capacity)\\
          & & & & in 'fl' fleet \\
          &  &	&	&	for the projection period\\
          & fl.met\_proj.avg.yrs\textsuperscript{*}    &	vector &	any &	Historic years to calculate averages (in effshare and vcost) in 'fl' fleet and\\
          & & & & 'met' metier for the projection period\\
          & fl.met.stk\_proj.avg.yrs\textsuperscript{*} &	vector &	any &	Historic years to calculate averages (in landings.wt, discards.wt, landings.sel,\\
          & & & & discards.sel, alpha, beta and catch.q) in 'fl' fleet, 'met' metier and 'stk' stock
          & & & & for the projection period\\
        stks.data & &	list & number of stocks &	List with the name of the stocks and the following elements:\\
          & stk.unit &	numeric &	1 &	Number of units\\
          & stk.age  &	numeric &	1 &	Number of age clases\\
          & stk.age.min &	numeric &	1 &	Minimum age\\ 
          & stk.age.max &	numeric &	1 &	Maximum age\\
          & stk\_wt.flq\textsuperscript{*} & FLQuant & [na,ny(hist),1/nu(stock),1/ns,1/ni] & Weight at age. Only required if fl.met.stk\_landings.wt is not defined\\
          & stk\_n.flq\textsuperscript{*}  & FLQuant & [na,ny(hist),1/nu(stock),1/ns,1/ni] & Numbers at age in the population (for stocks modelled in numbers at age).\\
          & & & & Only required if Cobb-Douglas parameters are not defined\\
          & stk\_gB.flq\textsuperscript{*} & FLQuant & [na,ny(hist),1/nu(stock),1/ns,1/ni] & Biomass growth (for stocks modelled in biomass).\\
          & & & & Only required if Cobb-Douglas parameters are not defined\\
        \hline
      \end{tabular}
      
      \begin{tablenotes}
        % \small
        \item na: number of age (from min.age to max.age)
        \item ny(hist): number of historic years (from first.yr to proj.yr-1)
        \item ny: number of years (from first.yr to last.yr)
        \item 1/nu(stock): 1 or number of units of the stock
        \item 1/ns: 1 or number of seasons
        \item 1/ni:  1 or number of iterations
      \end{tablenotes}
      
    \end{threeparttable}
  \end{footnotesize}
  %\end{center}

\end{table}	



%TABLE 5
\begin{table}[!ht]

  %\begin{center}
  \centering
  \begin{footnotesize}
    
    \caption{Description of the arguments of the function \texttt{create.indices.data}. 
      In the table we assume that \texttt{stk} is the name of the stock and \texttt{ind} the name of the index. 
      The arguments with \textsuperscript{*} are optional arguments.}
    
    \label{tb:A4.table5}
    
    \begin{threeparttable}
    
      \begin{tabular}{lllll} %{c|c}
        \hline 
        Argument & & class & Dimension & Definition\\
        \hline
        yrs & & vector & 3 &	c( first.yr, proj.yr, last.yr)\\
          & first.yr & numeric & 1 & First year of simulation\\
          & proj.yr  & numeric & 1 & First year of projection\\
          & last.yr  & numeric & 1 & Last year of projection\\
        ni & & numeric &	1 &	Number of iterations\\
        ns & & numeric &	1 &	Number of seasons\\
        stks.data & &	list & number of stocks &	List with the name of the stocks with indices and the following elements:\\
          & stk.unit     & numeric   & 1 & Number of units\\
          & stk.age      & numeric   & 1 & Number of age classess\\ 
          & stk\_indices & character & 1 & Name of indices for the stock 'stk'\\
          & stk\_ind\_type\textsuperscript{*} & character & 1 & Type of index\\
          & stk\_ind\_distribution\textsuperscript{*} & character & 1 & Name of the statistical distribution of the 'ind' index values for stock 'stk'\\
          & stk\_ind\_index.flq &	FLQuant &	[na,ny,1/nu(stock),1/ns,1/ni] &	Historical index data for index 'ind' of stock 'stk'\\
          &  stk\_ind\_index.var.flq\textsuperscript{*} &	FLQuant &	[na,ny,1/nu(stock),1/ns,1/ni] &	Variability in 'ind' index of stock 'stk'\\
          & stk\_ind\_index.q.flq\textsuperscript{*} &	FLQuant &	[na,ny,1/nu(stock),1/ns,1/ni] &	Catchability for 'ind' index of stock 'stk'\\
          & stk\_ind\_catch.n.flq\textsuperscript{*} &	FLQuant &	[na,ny,1/nu(stock),1/ns,1/ni] &	Catch at age in numbers for 'ind' index of stock 'stk'\\
          & stk\_ind\_catch.wt.flq\textsuperscript{*} &	FLQuant &	[na,ny,1/nu(stock),1/ns,1/ni] &	Mean weight at age in the catch for 'ind' index of stock 'stk'\\
          & stk\_ind\_effort.flq\textsuperscript{*} &	FLQuant &	[na,ny,1/nu(stock),1/ns,1/ni] &	Effort for 'ind' index of stock 'stk'\\
          & stk\_ind\_sel.pattern.flq\textsuperscript{*} &	FLQuant &	[na,ny,1/nu(stock),1/ns,1/ni] &	Selection pattern for 'ind' index of stock 'stk'\\
          & stk\_ind\_range.min\textsuperscript{*} &	FLQuant &	[na,ny,1/nu(stock),1/ns,1/ni] &	Minimum age in 'ind' index of stock 'stk'\\
          & stk\_ind\_range.max\textsuperscript{*} &	FLQuant &	[na,ny,1/nu(stock),1/ns,1/ni] &	Maximum age in 'ind' index of stock 'stk'\\
          & stk\_ind\_range.plusgroup\textsuperscript{*} &	FLQuant &	[na,ny,1/nu(stock),1/ns,1/ni] &	Plusgroup age in 'ind' index of stock 'stk'\\
          & stk\_ind\_range.minyear\textsuperscript{*} &	FLQuant &	[na,ny,1/nu(stock),1/ns,1/ni] &	First year with 'ind' index data of stock 'stk'\\
          & stk\_ind\_range.maxyear\textsuperscript{*} &	FLQuant &	[na,ny,1/nu(stock),1/ns,1/ni] &	Last year with 'ind' index data of stock 'stk'\\
          & stk\_ind\_range.startf\textsuperscript{*} &	FLQuant &	[na,ny,1/nu(stock),1/ns,1/ni] &	Minimum age for calculating average fishing mortality for 'ind' index of stock 'stk'\\
          & stk\_ind\_range.endf\textsuperscript{*}   &	FLQuant &	[na,ny,1/nu(stock),1/ns,1/ni] &	Maximum age for calculating average fishing mortality for 'ind' index of stock 'stk'\\          
        \hline
      \end{tabular}
      
      \begin{tablenotes}
        % \small
        \item na: number of age (from min.age to max.age)
        % \item ny(hist): number of historic years (from first.yr to proj.yr-1)
        \item ny: number of years (from first.yr to last.yr)
        \item 1/nu(stock): 1 or number of units of the stock
        \item 1/ns: 1 or number of seasons
        \item 1/ni:  1 or number of iterations
      \end{tablenotes}
      
    \end{threeparttable}
  \end{footnotesize}
  %\end{center}

\end{table}

	

%TABLE 6
\begin{table}[!ht]

  %\begin{center}
  \centering
  \begin{footnotesize}
    
    \caption{Description of the arguments of the function \texttt{create.advice.data}. 
      In the table we assume that \texttt{stk} is the name of the stock. 
      The arguments with \textsuperscript{*} are optional arguments.}
    
    \label{tb:A4.table6}
    
    \begin{threeparttable}
    
      \begin{tabular}{lllll} %{c|c}
        \hline 
        Argument & & class & Dimension & Definition\\
        \hline
        yrs & & vector & 3 &	c( first.yr, proj.yr, last.yr)\\
          & first.yr & numeric & 1 & First year of simulation\\
          & proj.yr  & numeric & 1 & First year of projection\\
          & last.yr  & numeric & 1 & Last year of projection\\
        ni & & numeric &	1 &	Number of iterations\\
        ns & & numeric &	1 &	Number of seasons\\
        stks.data & &	list & number of stocks &	List with the name of the stocks with indices and the following elements:\\

          & stk\_advice.TAC.flq\textsuperscript{*} &	FLQuant &	[na,ny,1/nu(stock),1/ns,1/ni] &	TAC of the stock 'stk'\\
          & stk\_advice.TAE.flq\textsuperscript{*} &	FLQuant &	[na,ny,1/nu(stock),1/ns,1/ni] &	TAE of the stock 'stk'\\
          & stk\_advice.quota.share.flq\textsuperscript{*} &	FLQuant &	[na,ny,1/nu(stock),1/ns,1/ni] &	Quota share of the stock 'stk'\\
          & stk\_advice.avg.yrs\textsuperscript{*} &	FLQuant &	any &	Mean weight at age in the catch for 'ind' index of stock 'stk'\\
        fleets\textsuperscript{*} & & FLQuant &	 &	Only required if \texttt{stk\_advice.quota.share} is not specified.\\
         & & & & Can be the output of \texttt{create\_fleets\_FLBEIA} function\\
        \hline
      \end{tabular}
      
      \begin{tablenotes}
        % \small
        \item na: number of age (from min.age to max.age)
        % \item ny(hist): number of historic years (from first.yr to proj.yr-1)
        \item ny: number of years (from first.yr to last.yr)
        \item 1/nu(stock): 1 or number of units of the stock
        \item 1/ns: 1 or number of seasons
        \item 1/ni:  1 or number of iterations
      \end{tablenotes}
      
    \end{threeparttable}
  \end{footnotesize}
  %\end{center}

\end{table}



%TABLE 7
\begin{table}[!ht]

  %\begin{center}
  \centering
  \begin{footnotesize}
    
    \caption{Description of the arguments of the function \texttt{create.list.stks.flqa}. 
      In the table we assume that \texttt{stk} is the name of the stock.}
    
    \label{tb:A4.table7}
    
    \begin{tabular}{lllll} %{c|c}
      \hline 
      Argument & & class & Dimension & Definition\\
      \hline
      stks & & vector & number of stocks &	Name of all the stocks\\
      yrs & & vector & 3 &	c( first.yr, proj.yr, last.yr)\\
        & first.yr & numeric & 1 & First year of simulation\\
        & proj.yr  & numeric & 1 & First year of projection\\
        & last.yr  & numeric & 1 & Last year of projection\\
      ni & & numeric &	1 &	Number of iterations\\
      ns & & numeric &	1 &	Number of seasons\\
      list.stks.unit & & list & number of stocks &	List with the name of the stocks and each stock contains the number\\
       & & & & of units\\
      list.stks.age & & list & number of stocks &	List with the name of the stocks and each stock contains a vector \\
       & & & & with minimum age (min.age) and maximum age (max.age)\\
      \hline
    \end{tabular}
      
  \end{footnotesize}
  %\end{center}

\end{table}



%TABLE 8
\begin{table}[!ht]

  %\begin{center}
  \centering
  \begin{footnotesize}
    
    \caption{Description of the arguments of the function \texttt{create.list.stks.flq}. 
      In the table we assume that \texttt{stk} is the name of the stock.}
    
    \label{tb:A4.table8}
    
    \begin{tabular}{lllll} %{c|c}
      \hline 
      Argument & & class & Dimension & Definition\\
      \hline
      stks & & vector & number of stocks &	Name of all the stocks\\
      yrs & & vector & 3 &	c( first.yr, proj.yr, last.yr)\\
        & first.yr & numeric & 1 & First year of simulation\\
        & proj.yr  & numeric & 1 & First year of projection\\
        & last.yr  & numeric & 1 & Last year of projection\\
      ni & & numeric &	1 &	Number of iterations\\
      ns & & numeric &	1 &	Number of seasons\\
      list.stks.unit & & list & number of stocks &	List with the name of the stocks and\\
       & & & & each stock contains the number of units\\
      \hline
    \end{tabular}
      
  \end{footnotesize}
  %\end{center}

\end{table}



%TABLE 9
\begin{table}[!ht]

  %\begin{center}
  \centering
  \begin{footnotesize}
    
    \caption{Description of the arguments of the function \texttt{calculate.CDparam}. 
      In the table we assume that \texttt{stk} is the name of the stock. 
      The arguments with \textsuperscript{*} are optional arguments.}
    
    \label{tb:A4.table9}
    
    \begin{threeparttable}
    
      \begin{tabular}{lllll} %{c|c}
        \hline 
        Argument & class & Dimension & Definition\\
        \hline 
        stk.n      & &	FLQuant &	[na,ny(hist),1/nu(stock),1/ns,1/ni] &	Abundance in numbers at age\\
        landings.n & &	FLQuant &	[na,ny(hist),1/nu(stock),1/ns,1/ni] &	Landings in numbers at age\\
        discards.n & &	FLQuant &	[na,ny(hist),1/nu(stock),1/ns,1/ni] &	Discards in numbers at age\\
        effort     & &	FLQuant &	[1,ny(hist),1/nu(stock),1/ns,1/ni]  &	Effort\\
        effshare   & &	FLQuant &	[1,ny(hist),1/nu(stock),1/ns,1/ni]  &	Effort share\\
        age.min    & &	numeric &	1                                   &	Minimum age\\
        age.max    & &	numeric &	1                                   &	Maximum age\\
        flqa       & &	FLQuant &	[na,ny(hist),1/nu(stock),1/ns,1/ni] &	An FLQuant object\\
        flq        & &	FLQuant &	[1,ny(hist),1/nu(stock),1/ns,1/ni]  &	An FLQuant object\\
        largs\textsuperscript{*} & & list & 1 &	A list with extra optional arguments:\\
          & stk.gB & numeric & 1 & Surplus production (only for stocks in biomass)\\
        \hline
      \end{tabular}
      
      \begin{tablenotes}
        % \small
        \item na: number of age (from min.age to max.age)
        \item ny(hist): number of historic years (from first.yr to proj.yr-1)
        \item 1/nu(stock): 1 or number of units of the stock
        \item 1/ns: 1 or number of seasons
        \item 1/ni:  1 or number of iterations
      \end{tablenotes}
    
    \end{threeparttable}      
  \end{footnotesize}
  %\end{center}

\end{table}



\end{landscape}