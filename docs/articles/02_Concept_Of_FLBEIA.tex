
\section{The concept of \texttt{FLBEIA}} \label{sec:FLBEIAconcept}

  The simulation model is divided in two main blocks, the Operating Model (OM) and the Management Procedure Model (MPM). This division is part of 
  the requirements of the MSE approach, that is, the model includes a mathematical representations of the \textit{real world} (OM), the 
  \textit{observed world} (MPM) and the interactions between them.


\subsection{Operating model}

  The OM is the part of the model that simulates the real dynamics of the fishery system. It is divided in three components or operating models, 
  the biological, the fleets and the covariates operating model. It runs in seasonal time steps, and it projects the components in each time step. 
  Firstly, it updates the biological component, secondly the fleet component and finally the covariates component.

\paragraph{Biological component} \hspace{0pt} \smallskip

  The biological component simulates the population dynamics of the stocks. The number of populations is, in principle, unlimited. 
  The limitation could come from memory problems with \texttt{R} and/or the operating system. The stocks can be described as age structured
  populations or as biomass dynamics populations, since length structured populations models are not supported by the simulation algorithm. 
  Each stock can follow a different population dynamics model and is projected independently. It does not mean that they  cannot be interdependent 
  between them but the order in which  these biological components are updated has to be decided and it will affect the results obtained.

\paragraph{Fleet component} \hspace{0pt} \smallskip

  The fleet component simulates the behaviour and dynamics of the individual fleets. As the number of the stocks, the number of fleets is in 
  principle unlimited. The limitation could come from memory problems with \texttt{R} and/or the operating system used. The activity of the fleets is divided into metiers. The metiers are formed by trips that have the same catchability for all the stocks. 
  Fleet fishing effort and effort share among metiers are independently updated for each fleet in each season. Fleet catchability and/or capacity is updated annually, independently for each fleet, through capital dynamics according to its own economic performance. 

\paragraph{Covariates component} \hspace{0pt} \smallskip

  This part of the model incorporates all the variables that are not part of the biological or fleet components and that affect any of the operating model components or the management process. The number of covariates is, in principle, unlimited. The limitation could come from memory problems with \texttt{R} and/or the operating system used.

\paragraph{Links among and within components} \hspace{0pt} \smallskip

  The links within the OM components are not restricted by the general settings of the simulation model. Therefore, it is the user who decides 
  which are the links that should be included in the model. The possible links that can be included are:

  \begin{itemize}
  	\item The link within the \textsl{biological} component, where catch affects abundance.
  	\item The link within the \textsl{fleets} component, where fleet capacity affects fishing effort. 
  	\item The link between the \textsl{biological and fleets} components, where fishing effort and fish abundance affects catches. 
  \end{itemize}

 
\subsection{Management procedure model}

  The Management Procedure Model (MPM) is divided into 3 components: the observation, the assessment and the management advice. The observation 
  component produces the required data to run the assessment. Then, the assessment component is applied to those data to obtain the observed 
  populations. Finally, the management advice component produces a management advice based on the observed populations. MPM procedure is applied 
  yearly in the appropriate season of the year. Not necessarily in the last season, for example, it can be simulated as in the case of anchovy in 
  the Bay of Biscay, where management is applied from the mid-season of one year to the mid-season of the next year. Simulations with multi-annual 
  advice is also possible.

\paragraph{Observation component} \hspace{0pt} \smallskip

  The observation component generates the required objects to run the assessments. Three types of objects can be generated:
  
  \begin{itemize}
  	\item Stocks. 
  	\item Fleets. 
  	\item Abundance indices.
  \end{itemize}
  
  Stocks and abundance indices objects are generated independently, stock by stock, whereas fleets are observed jointly. 
  These objects are generated based on the variation that is introduced in the components of the OM. This variation can be due to:
  
  \begin{itemize}
  	\item Introducing uncertainty to the OM variables, or
  	\item adjusting the OM variables to the assessment model requirements which is going to be used in the next step (e.g. collapsing the dimensions -age, season,...), or
  		\item adjusting the OM variables to the legal conditionings (TACs, quotas, TAE, discards,...).
  \end{itemize}
 
\paragraph{Assessment component} \hspace{0pt} \smallskip

  Assessment models are applied on a stock by stock basis and they can vary from stock to stock.

\paragraph{Management advice component} \hspace{0pt} \smallskip

  The management advice component produces a set of indicators (determined by the user) useful for policy making.
  The management advice is produced based on the output obtained from the observation and assessment components.
  The advice is first applied at single stock level and after that it can be applied at fleet level. 
